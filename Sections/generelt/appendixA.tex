\chapter{Diary}\label{ch:appDiary}
\section*{Genral work related information} \newline
 Working hours 8:30 - 17:30 Monday to Friday.
\section*{Daily worktasks}
\begin{description}
\item[(17/01)] First day at CERN. Introduced to my new colleagues and project.
\item[(18/01)] Introduction to the software tools and enviorments used.  
\item[(19/01)] Setup enviorments for developing. 
\item[(20/01)] Going through the manual by Morten, and looked into cross compiling.
\item[(21/01)] Researching PetaLinux, and started using VIM for developing C programs. First 'Hello world' on the EMP.
\item[(22/01)] Weekend.
\item[(23/01)] Weekend.
\item[(24/01)] Day off for taking the online project exam.
\item[(25/01)] Introduced to the NFS filesystem.
\item[(26/01)] Looking into developed code for communcation with the EMCI, which reads and writes registers of the lpGBT.
\item[(27/01)] Learning of Makefiles. Beginning developing of a cpp program to enable the clock generator present on the EMP. 
\item[(28/01)] Continuing working with the program. 
\item[(29/01)] Weekend
\item[(30/01)] Weekend
\item[(31/01)] Debugging the program with help from Daniel. 
\item[(1/02)] Continuing debugging and working on the code
\item[(2/02)] Learning about the I2C switch present on the board. 
\item[(3/02)] Getting into AirHDL. Further understanding for lpGBT, registers, AXI and I2C 
\item[(4/02)] Remote PC stopped working. Begining a new task of devoloping a GUI in python.
\item[(5/02)] Weekend
\item[(6/02)] Weekend
\item[(7/02)] Looking into Python
\item[(8/02)] Python excercise - Develope a GUI in .py which use a function from a .cpp file. 
\item[(9/02)] Excersise finished. Researching a solution for communication between the remote PC and the EMP, in such way that the remote PC proccessing the GUI.
\item[(10/02)] Further reasearching, and coding for test different approaches. 
\item[(11/02)] Looked further into an API. Including how bindings in .py (marshalling) works, Sockets, researching of what a IPC is, an the RESTful API architecture.  
\item[(12/02)] Weekend
\item[(13/02)] Weekend
\item[(14/02)] Beginning the 'Docker-getting-started' exercise
\item[(15/02)] Finished 'Docker-getting-started', dived further into docker by following some certification tasks and questions.
\item[(16/02)] Read Specifications documents regarding the EMP and further reading of Mortens manual
\item[(17/02)] Remote PC working again. Starting developing code for EMP.
\item[(18/02)] Looking further into how to connect the GUI with the EMP. Succesfully etablished a connection via a TCP socket between a .py and a .cpp code on two different devices.  
\item[(19/02)] Weekend
\item[(20/02)] Weekend
\item[(21/02)] Implemented multi thredding in code to trying to acomplish full duplex communicaton between server/client.
\item[(22/02)] Further working on implementing multi threadding in code, and made it succesfully. The server/client can now send and recieve at the same time.  
\item[(23/02)] Preparing Daniel's leaving at CERN. In this regard I looked further into communication between the EMP and the EMCI. Furthermore, trying to establish a broad understanding of the elements involved in the project, but with a deep understanding within my fiels of work. In such way, that I, with the help from Vladimir can continue the project, in a painless way. Hopefully any questions that might be relevant for this project in the future, will at this time be asked.  
\item[(24/02)] Looked into bit-banging, and the emp-intr code. 
\item[(25/02)] Meeting today for the ATLAS DCS front-end (EMCI - Embedded Monitoring and Control Interface) and back-end (EMP - Embedded Monitoring Processor).
\item[(26/02)] Weekend
\item[(27/02)] Weekend
\item[(28/02)] Problems with the clock. Trying to debug, and asking Daniel for help.
\item[(29/02)] 
\item[(01/03)] 
\item[(02/03)] 
\item[(03/03)] 
\item[(04/03)] 
\item[(05/03)] 
\item[(06/03)] 
\item[(07/03)] 
\item[(08/03)] 
\item[(09/03)] 
\item[(10/03)] Clock configuration finally works! Found the issue, which was cause by writing/reading 16-bit instead of 8-bit.
\item[(11/03)] 

\end{description}