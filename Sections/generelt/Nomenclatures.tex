\makenomenclature
\addcontentsline{toc}{chapter}{Nomenclature}
%\addcontentsline{toc}{section}{Acronyms}

%% This will add the subgroups
%----------------------------------------------
\renewcommand\nomgroup[1]{%
  \item[\bfseries
  \ifstrequal{#1}{A}{\LARGE{Acronyms}}{%
  \ifstrequal{#1}{B}{\LARGE{Glossaries}}{}}%
]}
%----------------------------------------------

%% This will add the units
%----------------------------------------------
\newcommand{\nomunit}[1]{%
\renewcommand{\nomentryend}{\hspace*{\fill}#1}}
%----------------------------------------------

% \nomenclature[B, 01]{For.}{{Udd.} \nomunit{\newline \textit{Besk}}}
%%%
\nomenclature[A, 01]{\textbf{NFS}}{{Network File System} \nomunit{\newline \textit{A protocol for file sharing. Allows access of files over a network. }}}
\nomenclature[A, 02]{\textbf{IP}}{{Intellectual Property} \nomunit{\newline \textit{A standalone module that can be used in the FPGA}}}
\nomenclature[A, 03]{\textbf{XDC file:}}{{Xilinx Design Constraint file} \nomunit{\newline \textit{Defines a list of requirements for the design, in order to function correctly on the specific board}}}


%%%
\nomenclature[A, 04]{\textbf{VL+:}}{{Versatile Link+} \nomunit{\newline \textit{Used for high speed optical links }}}
\nomenclature[A, 05]{\textbf{EMP:}}{{Embedded Monitoring Processor}\nomunit{\newline \textit{\vspace{-0.3 cm}}} }
\nomenclature[A, 06]{\textbf{EMCI:}}{{Embedded Monitoring and Control Interface}\nomunit{\newline \textit{\vspace{-0.3 cm}}}}
\nomenclature[A, 08]{\textbf{SoC:}}{{System-on-chip} \nomunit{\newline \textit{A SoC that includes several microprocessors}}}

\nomenclature[A, 08.2]{\textbf{MPSoC:}}{{Multiprocessor system-on-chip} \nomunit{\newline \textit{A SoC that includes several microprocessors}}}
\nomenclature[A, 08]{\textbf{lpGBT:}}{{Low power Giga Bit Transceiver} \nomunit{\newline \textit{Radiation tolerant ASIC}}}
\nomenclature[A, 09]{\textbf{B2B:}}{{board-to-board connector} \nomunit{\newline \textit{Connection correlation between two boards }}}

\nomenclature[A, 10]{\textbf{PL:}}{{Programmable Logic} \nomunit{\newline \textit{The FPGA part}}}
\nomenclature[A, 11]{\textbf{PS:}}{{Processing Subsystem} \nomunit{\newline \textit{Executes the applications and system programs}}}
\nomenclature[A, 12]{\textbf{FE:}}{{Front-ends} \nomunit{\newline \textit{E.g. a sensor in the detector.}}}
\nomenclature[A, 14]{\textbf{MGT:}}{{Multi-Gigabit Transceiver} \nomunit{\newline \textit{Capable of operating at serial bit rates above 1 Gigabit/second. }}}
\nomenclature[A, 15]{\textbf{EMI:}}{{Electromagnetic interference} \nomunit{\newline \textit{(Unwanted noise in a circuit.)}}}
\nomenclature[A, 16]{\textbf{LVDS:}}{{Low-Voltage Differential Signaling} \nomunit{\newline \textit{ Digital interface for serial communication, over two cables. Reduces EMI compared to single-ended connections. }}}
\nomenclature[A, 18]{\textbf{VSC}}{{Visual Studio Code} \nomunit{\newline \textit{Source code editor, with support of several languages.\vspace{1.5 cm}}}} %%% \vspace{1 cm}


%%%


\nomenclature[B, 01]{\textbf{TTY Driver}}{{Teletype writer} \nomunit{\newline \textit{Provide an interface for the terminal. Controlling both the flow of data and the format}}}
\nomenclature[B, 02]{\textbf{U-boot}}{{} \nomunit{\newline \textit{Boot loader used in Linux based embedded systems. Both a first and second stage bootloader.}}}
\nomenclature[B, 03]{\textbf{firewalld}}{{} \nomunit{\newline \textit{Firewall management tool on Linux OS}}}
\nomenclature[B, 04]{\textbf{gpiochip<n>}}{{} \nomunit{\newline \textit{gpiochip can be thought of as an "GPIO controller" i.e. one that controls several GPIO's}}}
\printnomenclature









