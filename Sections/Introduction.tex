% \chapter{Introduction}\label{ch:introduction}
% Here is the introduction. The next chapter is chapter~\ref{ch:ch2label}.


% a new paragraph


\section{Examples}
You can also have examples in your document such as in example~\ref{ex:simple_example}.
\begin{example}{An Example of an Example}
  \label{ex:simple_example}
  Here is an example with some math
  \begin{equation}
    0 = \exp(i\pi)+1\ .
  \end{equation}
  You can adjust the colour and the line width in the {\tt macros.tex} file.
\end{example}

\section{Code example}
\begin{listing}[!ht]
\begin{minted}[
        frame=single,
        obeytabs=true,
        tabsize=1,
        linenos,
        numbersep=10pt,
        highlightlines={2},
        highlightcolor=newyellow]
        {c}
#include <stdio.h>
int main() {
   printf("Hello, World!"); /*printf() outputs the quoted string*/
   return 0;
}
\end{minted}
\caption{Hello World in C} \label{listing:2}
\end{listing}

asdasd asdasd  a asd 
\mint{html}|<h2>Code example here in text! <b>here</b></h2>|
asdasda 
asdasd



\chapter{Working at CERN}
\section{CERN as an organization}

CERN was established in 1954 by a multilateral intergovernmental treaty, better known as the CERN convention. 
It has four fundamental tasks:

\begin{itemize}
    \item \textbf{Research:} Seeking and finding answers about the universe
    \item \textbf{Technology:} Advancing the frontier of technology  
    \item \textbf{Collaboration:} Bringing nations together through science 
    \item \textbf{Education:} Training the scientists of tomorrow    
\end{itemize}

\noindent CERN is divided into four sectors.

\begin{itemize}
    \item Accelerator and technology
    \item Research and computing
    \item International Relations
    \item Finance and Human Resources
\end{itemize}

\noindent These sectors are further divided into eleven departments.

\begin{itemize}
    \item Site and Civil Engineering
    \item Engineering
    \item Theoretical Physics
    \item Accelerator Systems
    \item Information Technology
    \item Technology
    \item Beams
    \item Experimental Physics
    \item Industry, Procurement and Knowledge Transfer
    \item Human Resources
    \item Finance and Administrative Processes
\end{itemize}

%%%%% Something about diversity  

In 2011 CERN established the diversity program within the Department of Human Resources. The program has three key principles:

\begin{itemize}
    \item   I:   Appreciating differences
    \item  II:  Fostering equality
    \item III: Promoting collaboration 
\end{itemize}

\noindent The program was later renamed to diversity and inclusion program and advised the organization to work towards an optimum-diverse workforce in a global laboratory.
Including how to build creativity and innovative from the collaboration between diverse ideas, perspectives and approaches. Furthermore, how to create an inclusive work environment.
\newline\newline
CERN Code of Conduct: Diversity at CERN has been a core feature of the population since it was founded. CERN working towards an enviroment there includes peoples from all nationalities of its members. Unfortunatly danish people are a minority at CERN. It's hard to get danish students at CERN because of the conditions in Denmark. Derived from this problem CERN has etablished a Danish Student Program. 

To transfer knowledge beetween nations has a great importance, not only for CERN, but for the nations as well. To work abroad, and getting new knowledge and experience has high value for the nations. 


\section{Work areas}



\section{Involvement in internship}
In the internship my main purpose are developing of the EMP. The EMP is a processing platform targeting for applications in experiment for the HL-LHC. It serves as an interface between EMCI and the control room. The EMP is placed at non-radtion areas, where it serves as an optical link reciever. Further it supports connection with multiple VL+ to the front end. The embedded proccesing 

\chapter{Relevant work areas at internship}

Courses such at ..... are related in many ways to the work at CERN.

Knowledges about programming languages beforehand has great influence a
such as C, C++, Python, has all been well used to

\section*{Hardware used:}
\begin{description}
    \item[EMP]  Embedded Monitoring Processor
    \item[EMCI] aadsda asd ass
\end{description}

\section*{Software used:}
\begin{description}
    \item[Vivado] Embedded asd asda d
    \item[Vitis] aadsda asd ass
\end{description}

\section{Definitions:}

\begin{description}
    \item[Embedded Software Development] The developement of creating a software platform from the hardware platform and develop the application code using the embedded CPU.
    \item[Board Support Package (BSP)] 
    \item[Image]
    \item[Network File System (NFS)] \todo[]{ move to nomenclatur}
    \item[Intellectual Property (IP)] \todo[]{ move to nomenclatur}
    \item[AXI]
    \item[Virtual memory]
    \item[]
    
\end{description}

\section{PetaLinux}

PetaLinux is used for customization, building and deploying Embedded Linux solutions on Xilinx processing systems. It is an embedded Linux Software Development Kit to handle the FPGA-based system design on an SoC. In this case, it will be targeting the UltraScale+ Xilinx MPSoC with the firmware. 

PetaLinux 



\section{Problem analysis}
\section{Theory}
\section{Methods}
\section{Models}
\section{Solution suggestions?}
\section{Implementation}
\section{Tests}
\section{Conclusion}
\section{m.m.}

\chapter{Analysis of internship}
\section{Professionally}
\section{Work related}
\section{Social}