

----------------------
SKABELONER OG GODE RÅD
----------------------

I dette dokument er der samlet en masse skabeloner til at kopiere ind i dine LaTeX-filer samt gode råd til brug af LaTeX. Det er opdelt sektionsvis (billeder, tabeller mv.). Med store bogstaver er det angivet, hvor du skal give specifikke input til koden, fx hvilket billede du ønsker at inkludere. Det er ofte fulgt af et eksempel. God fornøjelse!

______________
¤¤ Billeder ¤¤
¯¯¯¯¯¯¯¯¯¯¯¯¯¯

% Enkelt billede:
\begin{figure}[htbp] % (alternativt [H])
	\centering
	\includegraphics[width=STØRRELSE\textwidth]{MAPPE/FIL}
	\caption{FIGURTEKST.}
	\label{fig:LABEL}
\end{figure}

% Eksempel:
\begin{figure}[H]
	\centering
	\includegraphics[width=0.80\textwidth]{billeder/sketch.pdf}
	\caption{Opbygning af model.}
	\label{fig:sketch}
\end{figure}

%2 billeder i subfigures, så der er f.eks. 4.20a og 4.20b:
\begin{figure}[H]
    \centering
    \begin{subfigure}{.5\textwidth}
        \centering
        \includegraphics[width=.95\linewidth]{billeder/2. Analyse/EQ/VLFD.PNG}
        \caption{Variabel lav-frekvens droop.}
        \label{fig:VLFD}
    \end{subfigure}%
    \begin{subfigure}{.5\textwidth}
        \centering
        \includegraphics[width=.95\linewidth]{billeder/2. Analyse/EQ/VLFS.PNG}
        \caption{Variabel lav-frekvens shelf.}
        \label{fig:VLFS}
    \end{subfigure}
    \caption{To metoder til variabel lav-frekvens equalizer.}
    \label{fig:ParametricEQVLF}
\end{figure}

% wrapfigure
\begin{wrapfigure}{r}{0.55\textwidth}
    \centering
    \includegraphics[width=0.50\textwidth]{billeder/2. Analyse/Speaker.png}
    \caption{Højttaler med bevægende membran.}
    \label{fig:højttaler}
    \vspace{-15pt}
\end{wrapfigure}

% 2 billeder side om side: pls ikke brug det her, det er ass i forhold til det ovenfor

% Eksempel:
\begin{figure}[H]
	\centering
	\begin{minipage}[b]{0.48\textwidth}
	\centering
	\includegraphics[width=1.00\textwidth]{billeder/billede1.jpg} % Venstre billede
	\end{minipage}
	\hfill
	\begin{minipage}[b]{0.48\textwidth}
	\centering
	\includegraphics[width=1.00\textwidth]{billeder/billede2.jpg} % Højre billede
	\end{minipage}
	\\ % Figurtekster og labels
	\begin{minipage}[t]{0.48\textwidth}
	\caption{Hvis jeg har brugt denne, betyder det at jeg er homoseksuel.} % Venstre figurtekst og label
	\label{fig:billede1}
	\end{minipage}
	\hfill
	\begin{minipage}[t]{0.48\textwidth}
	\caption{Brug subfigures i stedet for din abe.} % Højre figurtekst og label
	\label{fig:billede2}
	\end{minipage}
\end{figure}

% Princippet bag brug af minipages er forsøgt illustreret nedenfor. Koden danner 4 minipages ("bokse"). I de 2 øverste inkluderes billederne, og i de 2 nederste lægges figurtekster og labels. Et linjeskift (\\) adskiller de 2 sæt bokse. I eksemplet ovenfor reserveres 48 pct. af bredden til hver minipage. Det resterende "luft" strækkes mest muligt med \hfill. Inde i hver minipage tillades billedet at fylde 100 pct. af boksens bredde. 

%   |¯¯¯¯|        |¯¯¯¯|
%   |    |        |    |
%   |____| \hfill |____| \\
%   |¯¯¯¯|        |¯¯¯¯|
%   |    |        |    |
%   |____| \hfill |____|

% Princippet kan også benyttes til at have 3 eller flere billeder side om side - tilføj blot flere minipages. Husk at skrue på minipage-størrelsen (fx 0.30 ved 3 billeder), og inkluder to ekstra \hfill. Der skal fortsat kun være ét linjeskift, og billede-størrelsen inde i minipagen skal også fortsat være 100 pct. 

______________
¤¤ Tabeller ¤¤
¯¯¯¯¯¯¯¯¯¯¯¯¯¯

% Eksempel - klassisk tabel:
\begin{table}[H] 
	\centering 
	\begin{tabular}{|l|l|l|l|l|l|} % Afstem antal tegn og kolonner! (l for venstre, c for center, r for højre, | for lodret streg) 
		\hline 	% Vandret streg
					  & Mandag & Tirsdag & Onsdag    & Torsdag   & Fredag  \\ \hline 	% Linjeskift og vandret streg
		09:00 - 10:00 & Kemi   & Dansk   & Matematik & Gymnastik & Engelsk \\ \hline 
		10:00 - 11:00 & Tysk   & Fransk  & Biologi   & Metal     & Fysik   \\ \hline 
	\end{tabular} 
	\caption{Peters skoleskema uge 41.} 
	\label{tab:skoleskema} 
\end{table}

% Eksempel - læsevenlig og flot tabel:
\begin{table}[H]
	\centering
	\begin{tabular}{lccl}	% Afstem antal tegn og kolonner! (l for venstre, c for center, r for højre)
		\toprule
		Case & Bemanding & Rapporter & Noter \\\midrule
		1 & 4 & 13 &            \\
		2 & 3 & 9  & Se notits  \\
		3 & 5 & 12 &            \\
		\bottomrule
	\end{tabular}
	\caption{Overblik over cases.}
	\label{tab:cases}
\end{table}

% 2 tabeller side om side
\begin{table}[H]
    \caption{Resultat af impedans målinger.}
    \label{tab:accepttestImpedansJournal}
    \begin{subtable}[h]{0.50\textwidth}
        \centering
        \caption{Resultater ved måling af indgangsimpedans.}
        \begin{tabular}{|c|c|c|}
        \hline
        \rowcolor{gray!50} \multicolumn{3}{|c|}{\textbf{Indgangsimpedans}}\\
        \hline
        $V\_{I1}$  & ting & $\E{V}$\\
        \hline
        $V\_{I2}$ & ting & $\E{V}$\\
        \hline
       \end{tabular}
       \label{tab:accepttestInputImpedansJournal}
    \end{subtable}
    \hfill
    \begin{subtable}[h]{0.50\textwidth}
        \centering
        \caption{Resultater ved måling af udgangsimpedans.}
        \begin{tabular}{|c|c|c|}
        \hline
        \rowcolor{gray!50} \multicolumn{3}{|c|}{\textbf{Udgangsimpedans}}\\
        \hline
        $V\_{O1}$. & ting & $\E{V}$\\
        \hline
        $V\_{O2}$ & ting & $\E{V}$\\
        \hline
       \end{tabular}
       \label{tab:accepttestOutputImpedansJournal}
     \end{subtable}
\end{table}

% Longtable
\begin{longtable}{|c|c|c|c|c|c|}
    \caption{Resultat for test af krav T.1-3.}
    \label{tab:accepttestResultatBehandlingFrekvens}\\
        \hline
        \rowcolor{gray!50}
        \multicolumn{2}{|c|}{}& \multicolumn{2}{c|}{Min forstærkning $\E{dB}$}& \multicolumn{2}{c|}{Maks forstærkning $\E{dB}$} \\\hline
         & Krav & \multicolumn{2}{c|}{-12} & \multicolumn{2}{c|}{12}\\\cline{2-6}
       Volumen & Målt & \multicolumn{2}{c|}{$\approx-70$} &\multicolumn{2}{c|}{$\approx+29$}\\\cline{2-6}
        & Afvigelse & \multicolumn{2}{c|}{\color{Green}$-58$} & \multicolumn{2}{c|}{\color{Green}$17$}\\\hline
        
        \rowcolor{gray!50}
        \multicolumn{2}{|c|}{}& \multicolumn{2}{c|}{Knækfrekvenser $\E{Hz}$} & \multicolumn{2}{c|}{Forstærkning $\E{dB}$} \\\cline{3-6}
       \rowcolor{gray!50}\multicolumn{2}{|c|}{} & Nedre & Øvre & Forstærkning & Relativ til maks \\\hline
       
       & Krav & 1 & 20.000 &  & \\\cline{2-6}
        System  & Målt & N/A  & 39.810 &  &  \\\cline{2-6}
         & Afvigelse & N/A & \color{Green}$99,05\%$ &  &  \\\hline
         & Krav & 16 & 250 &  & -12\\\cline{2-6}
        Bas  & Målt & N/A  & 251,2 & 17,19 & -11,81 \\\cline{2-6}
         & Afvigelse & N/A & \color{Green}$0,48\%$ &  & \color{Green}$0,19$ \\\hline
         
         & Krav & 250 & 4.000 &  & -12\\\cline{2-6}
        Mid  & Målt & 251,2 & 4.169 & 17,58 & -11,42 \\\cline{2-6}
         & Afvigelse & \color{Green}$0,48\%$ & \color{Green}$4,23\%$ &  & \color{Red}$0,58$ \\\hline
         
         & Krav & 4.000 & 20.000 &  & -12\\\cline{2-6}
        Diskant  & Målt & 4.169 & 21.880 & 17,65 & -11,35 \\\cline{2-6}
         & Afvigelse & \color{Green}$4,23\%$ & \color{Red}$9,40\%$ &  & \color{Red}$0,65$ \\\hline
\end{longtable}

% Flet kolonner:
\multicolumn{ANTAL}{JUSTERING}{INDHOLD}

% Eksempel:
\multicolumn{5}{c}{Regioner}

_______________
¤¤ Matematik ¤¤
¯¯¯¯¯¯¯¯¯¯¯¯¯¯¯

% Almindelig ligning:
\begin{align}
		
	\label{eq:LABEL} % \nonumber fjerner nummeret (label bliver derved overflødigt)
\end{align}

% Eksempel - med definition af variable efterfølgende:
\begin{align}
	\Phi = \rho \cdot c_p \cdot q_v \cdot \Delta T
	\label{eq:varmeflux}
\end{align}

Hvor:
\begin{table}[H]
	\begin{tabular}{l|l}
	$\Phi$     & Varmestrøm [\si{W}] \\
	$\rho$ 	   & Luftens densitet [\si{kg/m^3}] \\
	$c_p$ 	   & Luftens specifikke varmefylde [\si{J/kgK}] \\
	$q_v$	   & Volumenstrøm [\si{m^3/s}] \\
	$\Delta T$ & Temperaturforskel [\si{K}]
	\end{tabular}
\end{table}

% Eksempel med 2 rækker:
\begin{align}
& x + 2 = 8 	\label{eq:lign1} \\ 	% Rækkerne venstrejusteres ved "&"
& 7 = y + 5 z 	\label{eq:lign2}
\end{align}

% Matematik udenfor aligns (brødtekst, tabeller, figurtekster mv.):
\si{ENHED}
\SI{TAL}{ENHED}
$SPECIALTEGN$

% Eksempler:
\si{m^3}
\SI{9,82}{m/s^2}
$\alpha$

__________
¤¤ Kemi ¤¤
¯¯¯¯¯¯¯¯¯¯

% Kemiske formler:
\ce{FORMEL}

% Eksempler:
\ce{CO2}
\ce{Fe2O3}
\ce{HCO3-}

% RS-sætninger:
\rsphrase{NUMMER}

% Eksempel:
R1: \rsphrase{R1}

____________
¤¤ Labels ¤¤
¯¯¯¯¯¯¯¯¯¯¯¯

% Varianter:
\label{eq:...} 		(Ligning)
\label{fig:...} 	(Figur)
\label{tab:...} 	(Tabel)

% Det anbefales at starte alle labels med forkortelser, der repræsenterer objekttypen (fig, tab, eq, sec)

________________________
¤¤ Interne referencer ¤¤
¯¯¯¯¯¯¯¯¯¯¯¯¯¯¯¯¯¯¯¯¯¯¯¯

% Varianter:
\ref{LABEL} 		“4.3”
\vref{LABEL} 		“4.3 on page 31”
\eqref{LABEL} 		“(4.3)”

_________________________
¤¤ Eksterne referencer ¤¤
¯¯¯¯¯¯¯¯¯¯¯¯¯¯¯¯¯¯¯¯¯¯¯¯¯

% Varianter:
\citep{LABEL} 		(Passiv)
\citet{LABEL} 		(Aktiv)

% Tilføj yderligere information:
\citep[ SIDE, AFSNIT, KAPITEL MV.]{LABEL}

% Eksempler:
\citep[ s. 58]{fysikbog} 		->		[Efternavn, År, s. 58]
\citep[ kap. 7]{fysikbog} 		->		[Efternavn, År, kap. 7]

% NB! Start med et mellemrum i [ ]!

_______________
¤¤ Bilags-CD ¤¤
¯¯¯¯¯¯¯¯¯¯¯¯¯¯¯

% Henvisning til bilags-CD:
\citep[ FILNAVN]{cd}

% I litteratur.bib-filen er der lavet en kilde, som optræder som henvisning til bilags-CD'en. Udskift blot gruppenavn.

_____________
¤¤ Genveje ¤¤
¯¯¯¯¯¯¯¯¯¯¯¯¯

% Genveje kodet nederst i preamble:
$\decC$ 		->		°C 		(ˆ{\circ}\text{C})
$\dec$ 			->		° 		(ˆ{\circ})
\m 				->		· 		(\cdot)



\begin{minted}{Type}

\end{minted}

_____________________
¤¤ Fodnote i tabel ¤¤
¯¯¯¯¯¯¯¯¯¯¯¯¯¯¯¯¯¯¯¯¯
Tekst i tabel$^x$

%hvor:
%x er det tal som der skal stå i fodnoten

%det nedenunder skal stå lige efter tabeller
\addtocounter{footnote}{1}
\footnotetext{Fodnote.}